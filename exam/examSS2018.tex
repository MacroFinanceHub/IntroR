\documentclass[]{article}
\usepackage{lmodern}
\usepackage{amssymb,amsmath}
\usepackage{ifxetex,ifluatex}
\usepackage{fixltx2e} % provides \textsubscript
\ifnum 0\ifxetex 1\fi\ifluatex 1\fi=0 % if pdftex
  \usepackage[T1]{fontenc}
  \usepackage[utf8]{inputenc}
\else % if luatex or xelatex
  \ifxetex
    \usepackage{mathspec}
  \else
    \usepackage{fontspec}
  \fi
  \defaultfontfeatures{Ligatures=TeX,Scale=MatchLowercase}
\fi
% use upquote if available, for straight quotes in verbatim environments
\IfFileExists{upquote.sty}{\usepackage{upquote}}{}
% use microtype if available
\IfFileExists{microtype.sty}{%
\usepackage{microtype}
\UseMicrotypeSet[protrusion]{basicmath} % disable protrusion for tt fonts
}{}
\usepackage[margin=1in]{geometry}
\usepackage{hyperref}
\hypersetup{unicode=true,
            pdftitle={Introduction to R},
            pdfauthor={Student Name, Student ID, StudentName@uni-muenster.de},
            pdfborder={0 0 0},
            breaklinks=true}
\urlstyle{same}  % don't use monospace font for urls
\usepackage{color}
\usepackage{fancyvrb}
\newcommand{\VerbBar}{|}
\newcommand{\VERB}{\Verb[commandchars=\\\{\}]}
\DefineVerbatimEnvironment{Highlighting}{Verbatim}{commandchars=\\\{\}}
% Add ',fontsize=\small' for more characters per line
\usepackage{framed}
\definecolor{shadecolor}{RGB}{248,248,248}
\newenvironment{Shaded}{\begin{snugshade}}{\end{snugshade}}
\newcommand{\KeywordTok}[1]{\textcolor[rgb]{0.13,0.29,0.53}{\textbf{#1}}}
\newcommand{\DataTypeTok}[1]{\textcolor[rgb]{0.13,0.29,0.53}{#1}}
\newcommand{\DecValTok}[1]{\textcolor[rgb]{0.00,0.00,0.81}{#1}}
\newcommand{\BaseNTok}[1]{\textcolor[rgb]{0.00,0.00,0.81}{#1}}
\newcommand{\FloatTok}[1]{\textcolor[rgb]{0.00,0.00,0.81}{#1}}
\newcommand{\ConstantTok}[1]{\textcolor[rgb]{0.00,0.00,0.00}{#1}}
\newcommand{\CharTok}[1]{\textcolor[rgb]{0.31,0.60,0.02}{#1}}
\newcommand{\SpecialCharTok}[1]{\textcolor[rgb]{0.00,0.00,0.00}{#1}}
\newcommand{\StringTok}[1]{\textcolor[rgb]{0.31,0.60,0.02}{#1}}
\newcommand{\VerbatimStringTok}[1]{\textcolor[rgb]{0.31,0.60,0.02}{#1}}
\newcommand{\SpecialStringTok}[1]{\textcolor[rgb]{0.31,0.60,0.02}{#1}}
\newcommand{\ImportTok}[1]{#1}
\newcommand{\CommentTok}[1]{\textcolor[rgb]{0.56,0.35,0.01}{\textit{#1}}}
\newcommand{\DocumentationTok}[1]{\textcolor[rgb]{0.56,0.35,0.01}{\textbf{\textit{#1}}}}
\newcommand{\AnnotationTok}[1]{\textcolor[rgb]{0.56,0.35,0.01}{\textbf{\textit{#1}}}}
\newcommand{\CommentVarTok}[1]{\textcolor[rgb]{0.56,0.35,0.01}{\textbf{\textit{#1}}}}
\newcommand{\OtherTok}[1]{\textcolor[rgb]{0.56,0.35,0.01}{#1}}
\newcommand{\FunctionTok}[1]{\textcolor[rgb]{0.00,0.00,0.00}{#1}}
\newcommand{\VariableTok}[1]{\textcolor[rgb]{0.00,0.00,0.00}{#1}}
\newcommand{\ControlFlowTok}[1]{\textcolor[rgb]{0.13,0.29,0.53}{\textbf{#1}}}
\newcommand{\OperatorTok}[1]{\textcolor[rgb]{0.81,0.36,0.00}{\textbf{#1}}}
\newcommand{\BuiltInTok}[1]{#1}
\newcommand{\ExtensionTok}[1]{#1}
\newcommand{\PreprocessorTok}[1]{\textcolor[rgb]{0.56,0.35,0.01}{\textit{#1}}}
\newcommand{\AttributeTok}[1]{\textcolor[rgb]{0.77,0.63,0.00}{#1}}
\newcommand{\RegionMarkerTok}[1]{#1}
\newcommand{\InformationTok}[1]{\textcolor[rgb]{0.56,0.35,0.01}{\textbf{\textit{#1}}}}
\newcommand{\WarningTok}[1]{\textcolor[rgb]{0.56,0.35,0.01}{\textbf{\textit{#1}}}}
\newcommand{\AlertTok}[1]{\textcolor[rgb]{0.94,0.16,0.16}{#1}}
\newcommand{\ErrorTok}[1]{\textcolor[rgb]{0.64,0.00,0.00}{\textbf{#1}}}
\newcommand{\NormalTok}[1]{#1}
\usepackage{graphicx,grffile}
\makeatletter
\def\maxwidth{\ifdim\Gin@nat@width>\linewidth\linewidth\else\Gin@nat@width\fi}
\def\maxheight{\ifdim\Gin@nat@height>\textheight\textheight\else\Gin@nat@height\fi}
\makeatother
% Scale images if necessary, so that they will not overflow the page
% margins by default, and it is still possible to overwrite the defaults
% using explicit options in \includegraphics[width, height, ...]{}
\setkeys{Gin}{width=\maxwidth,height=\maxheight,keepaspectratio}
\IfFileExists{parskip.sty}{%
\usepackage{parskip}
}{% else
\setlength{\parindent}{0pt}
\setlength{\parskip}{6pt plus 2pt minus 1pt}
}
\setlength{\emergencystretch}{3em}  % prevent overfull lines
\providecommand{\tightlist}{%
  \setlength{\itemsep}{0pt}\setlength{\parskip}{0pt}}
\setcounter{secnumdepth}{5}
% Redefines (sub)paragraphs to behave more like sections
\ifx\paragraph\undefined\else
\let\oldparagraph\paragraph
\renewcommand{\paragraph}[1]{\oldparagraph{#1}\mbox{}}
\fi
\ifx\subparagraph\undefined\else
\let\oldsubparagraph\subparagraph
\renewcommand{\subparagraph}[1]{\oldsubparagraph{#1}\mbox{}}
\fi

%%% Use protect on footnotes to avoid problems with footnotes in titles
\let\rmarkdownfootnote\footnote%
\def\footnote{\protect\rmarkdownfootnote}

%%% Change title format to be more compact
\usepackage{titling}

% Create subtitle command for use in maketitle
\newcommand{\subtitle}[1]{
  \posttitle{
    \begin{center}\large#1\end{center}
    }
}

\setlength{\droptitle}{-2em}
  \title{Introduction to R}
  \pretitle{\vspace{\droptitle}\centering\huge}
  \posttitle{\par}
\subtitle{Exam Summer Term 2018}
  \author{Student Name, Student ID,
\href{mailto:StudentName@uni-muenster.de}{\nolinkurl{StudentName@uni-muenster.de}}}
  \preauthor{\centering\large\emph}
  \postauthor{\par}
  \date{}
  \predate{}\postdate{}


\begin{document}
\maketitle

{
\setcounter{tocdepth}{2}
\tableofcontents
}
\begin{center}\rule{0.5\linewidth}{\linethickness}\end{center}

\begin{itemize}
\item
  Answer \textbf{7} out of \textbf{10} of the following exercises in
  either German or English.
\item
  Hand in your solutions before Friday, 13 April 2018 at 10 am.
\item
  It is advised to regularly check the learnweb and your emails in case
  of urgent updates.
\item
  Please sent your solutions files to
  \href{mailto:willi.mutschler@wiwi.uni-muenster.de?subject=Exam\%20Intro\%20R}{Willi
  Mutschler}. We will confirm the receipt of your work also by email.
\item
  The solution files should contain your executable and commented script
  file or preferably a R Notebook.
\item
  You may use \emph{any} available R package you find fit to solve the
  exercise.
\item
  Please label your axes and title in your plots.
\item
  \textbf{All students must work on their own.}
\end{itemize}

\begin{center}\rule{0.5\linewidth}{\linethickness}\end{center}

\pagebreak

\section{Linear Equation}\label{linear-equation}

Solve the linear equation \(A\cdot x = b\) with \[
A = \begin{pmatrix} 1 & 1 & 1 & 3\\ 2 & 5 & 8 & 6\\ 4 & 3 & 7 & 9\\ 3 & 6 & 5 & 1 \end{pmatrix} \text{ and }
b = \begin{pmatrix} 1 \\ 4 \\ 3 \\ 5 \end{pmatrix}
\]

\textbf{\emph{Solution:}}

and compute the inverse as well as Eigenvalues of \(A\).

\textbf{\emph{Solution:}}

\pagebreak

\begin{center}\rule{0.5\linewidth}{\linethickness}\end{center}

\section{Functions}\label{functions}

\begin{enumerate}
\def\labelenumi{\arabic{enumi}.}
\tightlist
\item
  Write a function \texttt{psum(n,a)} that computes \[
  s_{n,a} := \sum_{k=0}^n \frac{k^a}{k^a+1}
  \] for any natural number \(n \in \{1,2,...\}\) and any \(a > 0\).
\end{enumerate}

\textbf{\emph{Solution:}}

\begin{enumerate}
\def\labelenumi{\arabic{enumi}.}
\setcounter{enumi}{1}
\tightlist
\item
  Write a function \texttt{mymatrix(n)} that returns a \(n \times n\)
  matrix such that: the first and last row as well as the first and last
  column contain only ones, whereas the remaining values are zero, e.g.
  \[
  \begin{pmatrix} 1 & 1 & 1 & 1\\ 1&0&0&1\\1&0&0&1\\1 & 1 & 1 & 1
  \end{pmatrix}
  \]
\end{enumerate}

\textbf{\emph{Solution:}}

\pagebreak

\begin{center}\rule{0.5\linewidth}{\linethickness}\end{center}

\section{Passenger numbers}\label{passenger-numbers}

The file \textbf{apass.csv} contains monthly data on passenger numbers
of US airlines from January 1949 to December 1959.

\begin{enumerate}
\def\labelenumi{\arabic{enumi}.}
\tightlist
\item
  Read the data into a data frame.
\end{enumerate}

\textbf{\emph{Solution:}}

\begin{enumerate}
\def\labelenumi{\arabic{enumi}.}
\setcounter{enumi}{1}
\tightlist
\item
  Create a vector that contains the corresponding dates. Add this vector
  to your data frame.
\end{enumerate}

\textbf{\emph{Solution:}}

\begin{enumerate}
\def\labelenumi{\arabic{enumi}.}
\setcounter{enumi}{2}
\tightlist
\item
  Plot the passenger numbers against the date vector.
\end{enumerate}

\textbf{\emph{Solution:}}

\begin{enumerate}
\def\labelenumi{\arabic{enumi}.}
\setcounter{enumi}{3}
\tightlist
\item
  Calculate the mean passenger numbers for each month. Plot the means as
  a bar chart.
\end{enumerate}

\textbf{\emph{Solution:}}

\pagebreak

\begin{center}\rule{0.5\linewidth}{\linethickness}\end{center}

\section{Porsche}\label{porsche}

The file \textbf{Porsche911.csv} contains data on Porsche 911 cars,
\texttt{name} is an id for the owner, \texttt{loc} is the location,
\texttt{age} is the age of the car, \texttt{TKM} is the mileage in
thousands kilometers and \texttt{price} the current price listed on an
internet plattform for used cars in thousand Euros.

\begin{enumerate}
\def\labelenumi{\arabic{enumi}.}
\tightlist
\item
  Read in the data into a data frame and compute key descriptive
  statistics (mean, standard deviation, smallest and largest values,
  quartiles, covariance and correlation) for the variables \texttt{age},
  \texttt{TKM} and \texttt{price}.
\end{enumerate}

\textbf{\emph{Solution:}}

\begin{enumerate}
\def\labelenumi{\arabic{enumi}.}
\setcounter{enumi}{1}
\tightlist
\item
  Plot boxplots for \texttt{age}, \texttt{TKM} and \texttt{price}.
\end{enumerate}

\textbf{\emph{Solution:}}

\begin{enumerate}
\def\labelenumi{\arabic{enumi}.}
\setcounter{enumi}{2}
\tightlist
\item
  Generate the empirical cumulative distribution function as well as
  histograms for each of the variables \texttt{age}, \texttt{TKM} and
  \texttt{price}.
\end{enumerate}

\textbf{\emph{Solution:}}

\pagebreak

\begin{center}\rule{0.5\linewidth}{\linethickness}\end{center}

\section{Law of large numbers}\label{law-of-large-numbers}

Let \(X_1, X_2,...\) be a sequence \(X_1,X_2,...\) from an \(AR(1)\)
process: \[
\left( X_{i}-\mu \right) =\rho \left( X_{i-1}-\mu \right) +\varepsilon _{i}
\] where \(\varepsilon _{i}\) is uniformly distributed on the interval
\([-1,1]\) and \(|\rho |<1\). Define the sequence of random variables \[
\bar{X}_{n}=\frac{1}{n}\sum_{i=1}^{n}X_{i}.
\] The weak law of large numbers states that the sample average
\(\bar{X}_{n}\) converges in probability towards the expected value
\(\mu\) when \(n\) tends to infinity.

Show by simulation that the law of large numbers holds despite the
intertemporal dependence in \(X\). In particular, show the convergence
by means of an appropriate plot. Hint: You may set the parameters to
e.g. \(\rho=0.5\) and \(\mu=2\) (or any other value you find fit.)

\textbf{\emph{Solution:}}

\pagebreak

\begin{center}\rule{0.5\linewidth}{\linethickness}\end{center}

\section{Limits of maxima}\label{limits-of-maxima}

Let \(X_{1}, X_{2},...\) be an i.i.d. sequence of random variables
uniformly distributed on the interval \([0,1]\). Define the random
variables \[
M_{n}=\max_{i=1,\ldots ,n}X_{i}
\] and its normalized version \(\overline{M}_{n}=(M_{n}-1)\cdot n\). One
can show that the limit distribution of \(\overline{M}_{n}\) is the
Weibull distribution with density \(\exp \left( x\right)\). Write an R
program to illustrate that \(\overline{M}_{n}\) converges in
distribution. To this end, set \(n=100\) and \(R=1000\) and consider the
\(R \times n\) matrix with uniformly distributed random variables
\(X_i\). Show the convergence by means of an appropriate plot.

Note: Try to avoid using loops (use e.g.~apply instead). You will not
get full points if you use a loop.

\textbf{\emph{Solution:}}

\pagebreak

\begin{center}\rule{0.5\linewidth}{\linethickness}\end{center}

\section{Student teacher ratio}\label{student-teacher-ratio}

Load the dataset \textbf{caschool.csv} into the object
\texttt{caschool}. This dataset is discussed in great detail in the
textbook of Stock and Watson.

\begin{enumerate}
\def\labelenumi{\arabic{enumi}.}
\tightlist
\item
  Make the following variables accessible:
\end{enumerate}

\begin{itemize}
\item
  test score \texttt{testscr}
\item
  student-teacher ratio \texttt{str}
\item
  percentage of English language learners \texttt{el\_pct}
\item
  expenditures per student \texttt{expn\_stu}
\end{itemize}

\textbf{\emph{Solution:}}

\begin{enumerate}
\def\labelenumi{\arabic{enumi}.}
\setcounter{enumi}{1}
\tightlist
\item
  Regress \texttt{testscr} on a constant and \texttt{str}. Assign the
  residuals of the regression into the variable \texttt{r1}. Now regress
  \texttt{testscr} on an intercept, \texttt{str}, \texttt{el\_pct} and
  \texttt{expn\_stu}. Put the residuals into the variable \texttt{r2}.
  Compute the sum of squared residuals for both regressions. Display
  \texttt{r1} and \texttt{r2} in one plot, where the points of
  \texttt{r2} are marked red.
\end{enumerate}

\textbf{\emph{Solution:}}

\begin{enumerate}
\def\labelenumi{\arabic{enumi}.}
\setcounter{enumi}{2}
\tightlist
\item
  Consider the regression of \texttt{testscr} on a constant,
  \texttt{str}, \texttt{elp\_ct} and \texttt{expn\_stu}. Predict the
  value of \texttt{testscr} for a school district with an average class
  size (\texttt{str}) of 25 students, a percentage of English learners
  (\texttt{el\_pct}) of 60\% and an average expenditures per student
  (\texttt{expn\_stu}) of 4000\$.
\end{enumerate}

\textbf{\emph{Solution:}}

\begin{enumerate}
\def\labelenumi{\arabic{enumi}.}
\setcounter{enumi}{3}
\tightlist
\item
  Reconsider the regression of \texttt{testscr} on a constant,
  \texttt{str}, \texttt{el\_pct} and \texttt{expn\_stu}. Compute the
  heteroscedastic robust standard errors.
\end{enumerate}

\textbf{\emph{Solution:}}

\begin{enumerate}
\def\labelenumi{\arabic{enumi}.}
\setcounter{enumi}{4}
\tightlist
\item
  Test the null hypothesis that the coefficients on \texttt{str} and
  \texttt{expn\_stu} both equal \(0\) and the coefficient on
  \texttt{el\_pct} equals \(-0.7\). Hint: Use the linearHypothesis
  function of the \texttt{car} package.
\end{enumerate}

\textbf{\emph{Solution:}}

\pagebreak

\begin{center}\rule{0.5\linewidth}{\linethickness}\end{center}

\section{Asymptotic normality}\label{asymptotic-normality}

Consider the multiple linear regression model \(y=X\beta +u\). In R,
generate the matrix \(X\) by executing the following commands:

\begin{Shaded}
\begin{Highlighting}[]
\KeywordTok{library}\NormalTok{(MASS)}
\NormalTok{X <-}\StringTok{ }\KeywordTok{cbind}\NormalTok{(}\DecValTok{1}\NormalTok{,}\KeywordTok{mvrnorm}\NormalTok{(}\DataTypeTok{n=}\DecValTok{100}\NormalTok{,}\KeywordTok{c}\NormalTok{(}\DecValTok{5}\NormalTok{,}\DecValTok{10}\NormalTok{),}\KeywordTok{matrix}\NormalTok{(}\KeywordTok{c}\NormalTok{(}\DecValTok{1}\NormalTok{,}\FloatTok{0.9}\NormalTok{,}\FloatTok{0.9}\NormalTok{,}\DecValTok{1}\NormalTok{),}\DecValTok{2}\NormalTok{,}\DecValTok{2}\NormalTok{)))}
\end{Highlighting}
\end{Shaded}

Assume that the true coefficient vector is \[
\beta =\left(
\begin{array}{c}
3 \\
2 \\
-1
\end{array}
\right)
\] and the error terms are i.i.d. uniformly distributed on the interval
\([-1,1]\). Hence, the assumption of normally distributed error terms is
violated.

\begin{enumerate}
\def\labelenumi{\arabic{enumi}.}
\tightlist
\item
  Write an R program that generates \(R=10000\) random samples of size
  \(n=100\) each. Generate an empty vector
  \texttt{Z\ \textless{}-\ rep(NA,R)}. For each sample
  \(i=1,\ldots ,R\), compute the OLS estimate \(\hat{\beta}\) of
  \(\beta\) and store the second component of \(\hat{\beta}\) in the
  \(i\)-th element of the vector \texttt{Z}.
\end{enumerate}

\textbf{\emph{Solution:}}

\begin{enumerate}
\def\labelenumi{\arabic{enumi}.}
\setcounter{enumi}{1}
\tightlist
\item
  Plot the histogram of \texttt{Z}. Compute the mean \(m\) and standard
  deviation \(s\) of \texttt{Z} and add the density of \(N(m,s)\) to the
  plot.
\end{enumerate}

\textbf{\emph{Solution:}}

\pagebreak

\begin{center}\rule{0.5\linewidth}{\linethickness}\end{center}

\section{Stochastic frontier
analysis}\label{stochastic-frontier-analysis}

Consider the Cobb-Douglas production function \[
y=Ax_{1}^{\alpha }x_{2}^{\beta }
\] By definition, the production function returns the maximal output for
given inputs, and actual production cannot be larger than \(y\). Due to
inefficiencies, actual production could be modeled (in logs) as \[
\ln y=\ln A+\alpha \ln x_{1}+\beta \ln x_{2}-u
\] where \(u\) is a \textbf{non-negative} random variable. Since other
disturbances (e.g.~measurement errors) can enter the production
function, it is more common to add another, \textbf{symmetrically
distributed}, disturbance term \(v\), \[
\ln y=\ln A+\alpha \ln x_{1}+\beta \ln x_{2}-u+v
\] Assume that \(u\) is exponentially (\(u \sim Exp(\lambda)\)) and
\(v\) normally (\(v\sim N(0,\sigma^{2})\)) distributed. One can show
that if \(u\) and \(v\) are independent then the density function of
\(\varepsilon =v-u\) is given by \[
f_{\varepsilon} (\varepsilon) =\lambda \exp\left(\lambda \varepsilon + \frac{1}{2} \lambda^{2} \sigma^{2}\right) \Phi \left( \frac{-\varepsilon}{\sigma} -\lambda \sigma \right)
\] where \(\Phi\) is the distribution function (\texttt{pnorm}) of
\(N(0,1)\) and \(\exp\) the exponential function (\texttt{exp}).

\begin{enumerate}
\def\labelenumi{\arabic{enumi}.}
\tightlist
\item
  Load the dataset \textbf{sfa.csv}. This dataset is an abbreviated
  version of table F7.2 of Greene, 2008. The original data appeared in
  Zellner and Revankar, \emph{Generalized Production Functions}, Review
  of Economic Studies, 36 (1969), 241-250.
\end{enumerate}

\textbf{\emph{Solution:}}

\begin{enumerate}
\def\labelenumi{\arabic{enumi}.}
\setcounter{enumi}{1}
\tightlist
\item
  Write an R program to estimate the parameters \(A\), \(\alpha\),
  \(\beta\), \(\lambda\) and \(\sigma\) by maximum likelihood on this
  dataset.
\end{enumerate}

\textbf{\emph{Solution:}}

\begin{enumerate}
\def\labelenumi{\arabic{enumi}.}
\setcounter{enumi}{2}
\tightlist
\item
  Compute the asymptotic standard errors.
\end{enumerate}

\textbf{\emph{Solution:}}

\pagebreak

\begin{center}\rule{0.5\linewidth}{\linethickness}\end{center}

\section{Variance estimation in
GARCH}\label{variance-estimation-in-garch}

When one considers an iid sample \(X_{1},\ldots ,X_{n}\) from
\(X\sim N(\mu ,\sigma ^{2})\) then one usually estimates the variance
\(\sigma^{2}\) using \[
\hat{\sigma}^{2}=\frac{1}{n-1}\sum_{i=1}^{n}(X_{i}-\bar{X})^{2}
\] The distribution of the normalized estimator for the variance is
given by: \[
\frac{\left( n-1\right) \hat{\sigma}^{2}}{\sigma ^{2}}\sim \chi _{n-1}^{2}
\] where \(\sigma^2\) is the true variance. Consider the case when the
observations are not iid but are stochastically dependent over time. To
this end, assume that \(X_{1},\ldots ,X_{n}\) is a time series generated
by a \(GARCH(1,1)\) process

\begin{align*}
X_{i} &\sim N(0,\sigma _{i}^{2}) \\
\sigma _{i}^{2} &=\omega +\alpha X_{i-1}^{2}+\beta \sigma _{i-1}^{2}
\end{align*}

with \(\omega =0.1\), \(\alpha =0.1\), \(\beta =0.85\) and sample size
equal to \(n=2500\). Show by simulations that the distribution of the
normalized estimator for the variance is not
\(\chi^2_{n-1}\)-distributed. Hint: The true unconditional variance of
this GARCH process is \(\sigma^{2}=2\).

\textbf{\emph{Solution:}}


\end{document}
